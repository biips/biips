\chapter{Codes}

\section{Simple example}
\label{miniex}

\lstset{language = C++, %
 basicstyle=\small\ttfamily, %
keywordstyle=\bfseries\color{RedViolet}, %
identifierstyle=,%
 commentstyle=\color{ForestGreen},%
stringstyle=\color{blue},%
%column=fixed,%
basewidth={0.5em,0.45em},%
numbers=left, numberfirstline=false, numberstyle=\tiny, stepnumber=10, numbersep=5pt}

\begin{lstlisting}
//                                               -*- C++ -*-
/*! \file t_Hmm1D_mini.cpp
 * \brief Biips minimal example
 * 
 * \author  $LastChangedBy: todeschi $
 * \date    $LastChangedDate: 2011-03-23 17:23:46 +0100 (mer. 23 mars 2011) $
 * \version $LastChangedRevision: 126 $
 * Id:      $Id: t_Hmm1D_mini.cpp 126 2011-03-23 16:23:46Z todeschi $
 *
 *
 * Biips example : Hidden Markov Model linear gaussian 1D
 * ======================================================
 * x[0] --> x[1] --> ... --> x[t] --> ...
 *           |                |
 *           v                v
 *          y[1]     ...     y[t]     ...
 *
 *          x[0] ~ Normal(mean_x0, var_x0)
 * x[t] | x[t-1] ~ Normal(x[t-1], var_x) for all t>0
 *   y[t] | x[t] ~ Normal(x[t], var_y)  for all t>0
 *
 */

#include "BiipsCore.hpp"
#include "BiipsBase.hpp"

int main(int argc, char* argv[])
{

  using namespace Biips;

 /*==================================*
  *                                  *
  * Instantiate the model with Biips *
  *                                  *
  *==================================*/

  // Define the data :
  //------------------
  // Final time :
  Size t_max = 20;
  // Initial mean of X[0]
  Scalar mean_x0_val = 0;
  // Variance of X[0] :
  Scalar var_x0_val = 1;
  // Variance of X[t] | X[t-1] :
  Scalar var_x_val = 1;
  // Variance of Y[t] | Y[t] :
  Scalar var_y_val = 0.5;

  // Declare FunctionTable and a DistributionTable :
  //------------------------------------------------
  FunctionTable func_tab;
  DistributionTable dist_tab;

  // Load the Base module :
  //-----------------------
  loadBaseModule(func_tab, dist_tab);

  // Declare the Graph object :
  //---------------------------
  Graph graph;

  // Add the constant nodes to the graph :
  //--------------------------------------
  NodeId mean_x0 = graph.AddConstantNode(MultiArray(mean_x0_val));
  NodeId var_x0 = graph.AddConstantNode(MultiArray(var_x0_val));
  NodeId var_x = graph.AddConstantNode(MultiArray(var_x_val));
  NodeId var_y = graph.AddConstantNode(MultiArray(var_y_val));

  // Create stochastic NodeId collections to store X and Y's components identifiers :
  //---------------------------------------------------------------------------------
  Types<NodeId>::Array x(t_max+1);
  Types<NodeId>::Array y(t_max);

  // Create a NodeId array to handle the parents of each stochastic node :
  //----------------------------------------------------------------------
  Types<NodeId>::Array params(2);

  // Add X_0 stochastic node to the graph :
  //---------------------------------------
  params[0] = mean_x0;
  params[1] = var_x0;
  x[0] = graph.AddStochasticNode(P_SCALAR_DIM, dist_tab["dnormvar"], params, false);

  // Add the other stochastic nodes to the graph :
  //----------------------------------------------
  for (Size t=1; t<t_max+1; ++t)
  {
    // Add X[t]
    params[0] = x[t-1];
    params[1] = var_x;
    x[t] = graph.AddStochasticNode(P_SCALAR_DIM, dist_tab["dnormvar"], params, false);

    // Add Y[t]
    params[0] = x[t];
    params[1] = var_y;
    y[t-1] = graph.AddStochasticNode(P_SCALAR_DIM, dist_tab["dnormvar"], params, true);
  }

  // Build the graph :
  //------------------
  graph.Build();

  // Generate data :
  //----------------
  // Declare a random number generator, initialized with an integer seed
  Rng my_rng(0);

  // Sample values according to the stochastic nodes prior distribution
  NodeValues gen_values = graph.SampleValues(&my_rng);

  // Set the observations values according to the generated values
  graph.SetObsValues(gen_values);


 /*==================================*
  *                                  *
  *     Run the SMC algorithm        *
  *                                  *
  *==================================*/

  // Define the number of particles :
  //---------------------------------
  Size nb_particles = 1000;

  // Declare the SMCSampler object :
  //--------------------------------
  SMCSampler sampler(nb_particles, &graph, &my_rng);
  sampler.SetResampleParams(SMC_RESAMPLE_STRATIFIED, 0.5);

  // Initialize the SMCSampler object :
  //-----------------------------------
  sampler.Initialize();

  // Declare and configure a ScalarAccumulator object :
  //---------------------------------------------------
  ScalarAccumulator stats_acc;
  stats_acc.AddFeature(MEAN);
  stats_acc.AddFeature(VARIANCE);

  // Declare Scalar arrays to store the posterior mean and variance estimates :
  //---------------------------------------------------------------------------
  Types<Scalar>::Array x_est_PF(t_max+1);
  Types<Scalar>::Array x_var_PF(t_max+1);

  // Iterate the SMC algorithm and extract summary statistics :
  //-----------------------------------------------------------
  for (Size t=0; t<t_max+1; ++t)
  {
    // Iterate the SMC algorithm : resample (if needed) and sample one node of the sequence
    sampler.Iterate();

    // Accumulate particles corresponding to the last updated node
    sampler.Accumulate(x[t], stats_acc);

    // extract summary statistics of the filtering density
    x_est_PF[t] = stats_acc.Mean();
    x_var_PF[t] = stats_acc.Variance();
  }

  return 0;

}
\end{lstlisting}
% 
% \chapter{Tests input files}
% \label{testsinput}
% \section{Test 1}
% 
% \begin{table*}[h!]
% \begin{tabular}{l|l}
% 20         & Final time : t\_max \\
% 0         &  Initial mean of x[0] : mu\_x0 \\
% 1        &   Variance of x[0] : sig\_x0 \\
% 1        &   Mean multiplicator of x[t] | x[t-1] : a\_x \\
% 0        &   Mean second member of x[t] | x[t-1] : b\_x \\
% 1        &   Variance of x[t] | x[t-1] : sig\_x \\
% 1         &  Mean multiplicator of y[t] | x[t] : a\_y \\
% 1         &  Variance of y[t] | x[t] : sig\_y \\
% 4589      &  Random number generator seed : rng\_seed \\
% 1000      &  Number of particles : N \\
% results.dat & Results data file \\
% plot.pdf   & Plot file \\
% gnuplot.plt & Gnuplot file
% \end{tabular}
% \caption{\texttt{data.dat}}
% \end{table*}
% 
% \section{Test 2}
% 
% \begin{table*}[h!]
% \begin{tabular}{l|l}
% 1                 & State x dimension : dim\_x \\
% 1                & Control u dimension : dim\_u \\
% 1                 & Observation y dimension : dim\_y \\
% 20                & Final time : t\_max \\
% 0                & Initial state mean vector : mu\_x0 \\
% 1                & Initial state covariance matrix : P0 \\
% 1                & State transition matrix : F \\
% 0                & Control transition matrix : B \\
% 0                & Control vector : u \\
% 1                & State covariance matrix : Q \\
% 1                & Observation transition matrix : H \\
% 1                & Observation covariance matrix : R \\
% 4589            &         Random number generator seed : rng\_seed   \\          
% 1000              &     Number of particles : N     \\          
% results\_1d.dat  & Results data file      \\             
% plot\_1d.pdf     & Plot file \\
% gnuplot\_1d.plt   & Gnuplot file
% \end{tabular}
% \caption{\texttt{data\_1d.dat}}
% \end{table*}
% 
% 
% \begin{table*}[h!]
% \begin{tabular}{l|l}
% 4   &   State x dimension : dim\_x\\
% 1  & Control u dimension : dim\_u\\
% 2  & Observation y dimension : dim\_y\\
% 10 &  Final time : t\_max \\
% 0  & Initial state mean vector : mu\_x0\\
% 0 & \\
% 1 & \\
% 1 & \\
% 1 0 0 0   &Initial state covariance matrix : P0\\
% 0 1 0 0 & \\
% 0 0 1 0 & \\
% 0 0 0 1 & \\
% 1 0 1 0  & State transition matrix : F\\
% 0 1 0 1 & \\
% 0 0 1 0 & \\
% 0 0 0 1&  \\
% 0  & Control transition matrix : B \\
% 0 & \\
% 0 & \\
% 0 & \\
% 0  & Control vector : u \\
% 1 0 0 0  & State covariance matrix : Q \\
% 0 1 0 0 & \\
% 0 0 1 0 & \\
% 0 0 0 1 & \\
% 1 0 0 0  & Observation transition matrix : H \\
% 0 1 0 0 & \\
% 1 0  & Observation covariance matrix : R \\
% 0 1 & \\
% 4535  & Random number generator seed : rng\_seed    \\
% 1000 &  Number of particles : N           \\    
% results\_4d.dat &  Results data file     \\              
% plot\_4d.pdf &  Plot file \\
% gnuplot\_4d.plt   & Gnuplot file
% \end{tabular}
% \caption{\texttt{data\_4d.dat}}
% \end{table*}
% 
% \section{Test 3}
% 
% \begin{table*}[h!]
% \begin{tabular}{l|l}
% 20   &   Final time : t\_max \\
% 2     &   Prior parameter alpha : alpha \\
% 2     &  Prior parameter beta : beta \\
% 10        &   Number of trials : n \\
% 4589       &          Random number generator seed : rng\_seed   \\
% 1000              &   Number of particles : N   \\            
% results\_beta.dat   &  Results data file  \\                 
% plot\_beta.pdf      &  Plot file \\
% gnuplot\_beta.plt   &  Gnuplot file
% \end{tabular}
% \caption{\texttt{data\_beta.dat}}
% \end{table*}


